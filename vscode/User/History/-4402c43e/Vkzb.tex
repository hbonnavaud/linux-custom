\section{Automated planning}\label{section:bg:planning}
%ALEX: Je l'appellerais plutôt "Automated Planning". Puis tu dis que Classical Planning (ou STRIPS) est quand les actions sont des effets deterministes.

The automated planning problem is frequently defined as the selection of actions in order to achieve a desired objective.
% ALEX Il faut le différencier du RL, où le but est semblable.(tu as un paragraphe dans ce style dans le rapport Parhéro que je t'ai passé)
% Dire que Planning dispose d'un modèle duquel on peut extraire, grÂce à la fonction de transition, un graphe d'atteignabilité.
% De ma compréhension, la seul différence c'est qu'n RL on va apprendre (on résuit une certaine erreur) les actions à effectuer alors qu'en automated planning on calcule directement l'action optimale.

This problem can be conceptualised as a graph of nodes, representing potential states, interconnected by edges, representing actions.
% ALEX : Ici tu dis que tu te focalises sur ce qu'on appelle Classical Planning, où les transitions dans le graphe orienté sont déterministes, i.e. pas de 'fils' OR.
% Ce serait aussi bien de faire le rapprochement avec STRIPS, que les lecteurs connaissent.

The Classical planning problem is solved by formulating a plan, which is in classical planning defined as a sequence of actions or edges that originates at the initial node and ends in one of the designated goal nodes.
In order to be represented as a classical planning model, which must satisfy three preconditions.
\begin{itemize}
    \item a finite set of states;
    \item deterministic transitions between states;
    \item full information about the initial state;
\end{itemize}

A classical planning problem can be translated as a directed graph, with the nodes representing states and the edges representing actions.
A transition is then directed from a source node (or state) towards a destination node (or next state).
A solution plan is thus defined as a sequence of transitions in this graph, leading from the initial node to another node that is an element of the goal nodes set, i.e. a linearly ordered finite sequence of actions.

\begin{definition}[Classical Planning Model]\label{definition:bg:plan:classical-planning-model}
    The classical planning model could then be defined by:
    \begin{itemize}
        \item A finite set of states $S$;
        \item An initial state $s_0 \in S$;
        \item A set of goal states $S_G \subseteq S$;
        \item A set of actions $A$;
        \item A state transition function $f: S \times A \rightarrow S: (s, a) \mapsto f(s, a) = s'$;
    \end{itemize}
\end{definition}
The state resulting from the application of the action $a$ to the given state $s$ is denoted $s'$.

The applicability of a fixed action $a$ in a state $s$ is defined as the existence of at least one target state $s'$ such that $f(s, a) = s'$.
We define the applicability domain as the subset $S_a$ such that, $f \mid S_a$ is injective.
Executing a sequence of applicable actions $[a0, \dots, an]$ onto a given state $s_0$ results in a chain of states such that $f(s_0, [a_0, \dots, a_n]) = [s_0, \dots, s_{n+1}]$, with $f(s_i, a_i) = s_{i+1}$, for $0 \leq i \leq n$, and $a_i$ an applicable operator in $s_i$.
We define the action $a_0$ as the \("\)no-operation\("\) action, such that $f(s_0, a_0) = s_0$.

A plan for a classical planning problem is constituted by a sequence of actions that are designed to achieve a goal
state from the initial state of the problem. % ALEX: Fais gaffe, tu l'as déjà dit plus haut. JE garderais cette jolie définition formelle et couperait en haut, ligne 10.

\begin{definition}[Classical Plan]\label{definition:classical-plan}
    A plan $p$ for a classical problem  $P = \langle S, s_0, S_G, A, f \rangle$ is an action sequence
    $p = \left[ a_0, \dots, a_n \right]$ that, once applied on the initial state, results in a sequence of states
    $[ s_0, \dots, s_n ]$, such that:
    \begin{itemize}
        \item all actions are applicable, and $s_{i+1} = f (s_i, a_i)$,
        \mbox{for $\,0 \le i \le n$.}
        \item the sequence terminates in a goal state: $s_{n+1} \in S_G$.
    \end{itemize}

    We sometimes use the writing $result(p, s_0) = s_{n+1}$ to indicate that the state resulting from the execution of
    $p$ on $s_0$ is $s_{n+1}$.
\end{definition}

The solution to a planning problem is defined as the shortest plan. % ALEX : Faux. Une solution est un plan. Une solution optimale est le plan le plus court.
However, in certain cases, the actions associated with a given problem are subject to a range of costs.
In such instances, the objective is to identify the plan, $p = \left[a_0, \dots , a_n\right]$, that minimises the total
cost, $cost(p) =  \sum_{i \in [0, n]}cost(a_i)$.
The planning model with costs is then defined as:
\begin{definition}[Classical planning model with costs]\label{definition:planning-cost}
    A planning model with costs $P_c$ consists of a planning model $P_c = \langle S, s_0, S_G, A, f \rangle$ along with
    a function \linebreak \hbox{$c: A \mapsto \mathbb{R}_0^+$} that maps each operator in the model to a non-negative
    cost.
\end{definition}
