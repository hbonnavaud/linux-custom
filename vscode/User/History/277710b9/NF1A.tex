\chapter{Related work}\label{chapter:related_works}

\section{Reinforcement learning} \label{section:related_works:rl}

Reinforcement learning (RL) has evolved significantly over the years, with numerous algorithms contributing to
its development.
Below is a chronological description of key milestones in RL algorithms, highlighting their contributions to the field.
The early foundation of RL is rooted in Dynamic Programming (DP)~\citep{bellman1954theory} (1954) methods.
The key idea behind DP is to solve Markov Decision Processes (MDPs) by recursively breaking down the problem using
the Bellman equation, which relates the value of a state to the values of subsequent states.
%
Following this, Policy Iteration~\citep{howard1960dynamic} (1960) and Value Iteration algorithms were developed.
Policy Iteration works by evaluating a given policy, improving it, and iterating the process until the optimal policy
is found.
On the other hand, Value Iteration computes the optimal value function through an iterative process, which can then
be used to derive the optimal policy.
%
Both methods were effective for small, discrete state spaces but struggled with large-scale or continuous domains.
In the 1970s, Monte Carlo methods became an important milestone in RL.
These methods estimate value functions by averaging the returns from multiple episodes, allowing the agent to learn
from actual experience rather than relying on theoretical models.
This approach was a key advancement, as it allowed for learning from real-world interactions, bypassing the need
for full knowledge of the environment.
Temporal Difference (TD) Learning~\citep{sutton1988learning} (1988) combines elements of both Monte Carlo and DP
methods, allowing agents to update their value function estimates based on both immediate rewards and the expected
future rewards from subsequent states.
This approach enabled agents to learn more efficiently from limited interactions with the environment,
overcoming some of the limitations of Monte Carlo methods.
%
Q-learning~\cite{watkins1989learning} (1989) is a model-free, off-policy RL algorithm that learns the optimal
action-value function, Q(s, a), which estimates the expected cumulative reward of taking action $a$ in state $s$.
Q-learning does not require a model of the environment and can learn the optimal policy purely from interacting
with the environment.
This algorithm has been fundamental in RL research and remains widely used in various applications today.
%
REINFORCE~\citep{williams1992simple} (1992), is a policy gradient method that computes the gradient of the expected
return with respect to the policy parameters, allowing for direct optimization.
This method is particularly useful in environments with large or continuous action spaces, where discretizing the
actions would be computationally infeasible.
%
Following Q-learning, SARSA (State-Action-Reward-State-Action)~\citep{rummery1994line} (1994) is an
on-policy~\ref{subsection:bg:rl:on-off-policy} alternative to Q-learning.
This difference makes SARSA more conservative and less prone to overestimation bias, which can be beneficial in
certain situations where a more cautious approach is needed.

In parallel, Actor-Critic methods where developed~\cite{barto1983neuronlike} (1983).
Actor-Critic methods are reinforcement learning algorithms that combine policy-based (Actor) and value-based (Critic)
learning.
The actor updates the policy using gradients provided by the critic, which estimates the q-value to reduce variance in
learning.
This framework enables learning in high-dimensional, continuous action spaces.
%
NAC~\citep{kakade2001natural} (2001) introduced the natural policy gradient, which improves stability by using the
Fisher information matrix to scale policy updates.
Unlike standard policy gradients, NAC follows a more efficient update direction, reducing variance and improving
convergence.
This method laid the foundation for more stable and efficient policy optimization in Actor-Critic methods.
%
DDPG~\citep{lillicrap2015continuous} (2015) extended Deterministic Policy Gradient (DPG) by integrating deep neural
networks, enabling learning in continuous action spaces.
Inspired by DQN, it introduced experience replay (to improve sample efficiency) and target networks
(to stabilize learning).
However, it suffered from overestimation bias and poor exploration.
%
A2C~\citep{mnih2016asynchronous} (2016) is a synchronous version of A3C, where multiple agents collect data in parallel,
improving sample efficiency.
It stabilizes training by removing asynchronous updates, making learning more predictable compared to A3C.
A2C remains simple and effective but lacks trust region constraints.
%
PPO\citep{schulman2017proximal} (2017) improved A2C by introducing a clipped surrogate objective, preventing excessively large
policy updates and improving training stability.
Unlike NAC and A2C, PPO doesn't require second-order gradients (making it computationally cheaper) and balances
exploration and exploitation more effectively.
It became a widely adopted baseline for policy optimization due to its simplicity and robustness.
%
SAC~\citep{haarnoja2018soft} (2018) addressed DDPG's instability by introducing entropy regularization, which encourages
exploration by maximizing reward while maintaining policy randomness.
It is off-policy, meaning it can reuse old experience, leading to better sample efficiency than PPO and A2C.
SAC excels in continuous control tasks, making it a major breakthrough in reinforcement learning.
%
TD3~\citep{fujimoto2018addressing} (2018) improved DDPG by fixing overestimation bias using clipped double Q-learning,
which takes the minimum of two Q-values to avoid optimistic value estimates.
It also introduced delayed policy updates and target smoothing, improving learning stability.
TD3 became the go-to choice for deterministic policy learning in continuous action spaces.
%
% Eventuellement rajouter A3C, TRPO, C51
% Value iteration ?
%
The field continues to evolve, with new methods being introduced regularly to enhance stability, efficiency, and
applicability across various domains.

\textbf{Goals in Reinforcement Learning (RL).}

Kaelbling~\cite{kaelbling1993learning} introduced one of the earliest works on goal-conditioned reinforcement learning,
emphasizing how policies can generalize across different goals.
Unlike traditional RL, which often learns a policy specific to a single task, this work laid the foundation for
multi-goal learning by treating goals as part of the state space.
This early formulation influenced later approaches that leveraged function approximation and hierarchical structures to
improve generalization.
%
Sutton et al.~\cite{sutton1999between} introduced the options framework, a key precursor to hierarchical RL, which is
particularly useful in goal-conditioned settings.
Options represent temporal abstractions—sub-policies that achieve intermediate goals—which enable efficient
long-horizon planning.
This work provided a theoretical basis for learning reusable skills, which later influenced structured exploration and
goal-directed behavior in RL.
%
Universal Value Function Approximators (UVFA)~\cite{schaul2015universal} introduced a way to generalize value functions
across different goals by learning a Q-function conditioned on goals.
Unlike standard RL, which learns a separate policy for each task, UVFA allows zero-shot generalization to unseen goals.
This laid the foundation for goal-conditioned RL by enabling policies that could achieve multiple goals efficiently.
%
Hindsight Experience Replay (HER)~\cite{andrychowicz2017hindsight} built on UVFA by introducing hindsight learning,
where failed attempts to reach a goal are relabelled as successes for training.
This drastically improved sample efficiency, as even unsuccessful trials provide useful learning signals.
HER enabled goal-conditioned RL in sparse reward environments, making it practical for real-world robotics and
high-dimensional tasks.
%
Visual Goal-Conditioned RL~\cite{nair2018visual} extended goal-conditioned RL to image-based tasks by introducing a
latent space representation of goals.
This allowed RL agents to set and reach goals directly in raw pixel spaces, enabling applications in vision-based
robotic manipulation.
The key advantage was making goal-conditioned RL more scalable to real-world environments.

Planning with Goal-Conditioned Policies~\cite{nasiriany2019planning} combined model-based planning with goal-conditioned
RL, addressing the challenge of long-horizon tasks.
Unlike previous approaches that relied solely on trial-and-error learning, it used self-supervised world models to
predict the best way to reach a goal.
This resulted in faster learning and better generalization to unseen goals.

\section{Hierarchical agents}

Many approaches use hierarchical agents to simplify RL training.

% Options-based Reinforcement Learning
\textbf{Learning abstract representations for planning.}
An alternative to crafting a hierarchy of learned policies is to rely on RL for producing ``lower level'' option
policies, and on some model of how these options affect the environment.
The aim is then to optimize a sequence of options, or skills, which are an abstraction of actions, in a global plan,
defined for an abstraction of states (which can either be learned or can be defined by the goal space).
The key to such approaches hence relies on how the model is built.
% MODEL-BASED RL
\citet{silver2017predictron} train a ``predictron'' which, for a given task, predicts $n$-step returns and long term
values from any state, using a network that builds a consistent internal representation (the state abstraction) of the
environment's dynamics and rewards.
Similarly, several approaches~\citep{ha2018world,schrittwieser2020mastering,hafner2020dream,hafner2021mastering} build
models that emulate the dynamics and rewards related to a task, and permit planning by simulating this surrogate model,
but without a hierarchy of options and for a single task.
In contrast, \citet{nasiriany2019planning} optimize a sequence of reachable intermediate goal states (represented in the
latent space of a variational auto-encoder on states) in order to reach a final goal (single task), using a pre-computed
reachability metric for a given goal-based lower-level policy.
\citet{parascandolo2020divide} optimize online a similar curriculum of sub goals between a starting state and a given
goal.
They implement a divide-and-conquer approach by building an AND/OR search tree.
Each node corresponds to a new sub goal in the sequence.
They explore this tree with a Monte Carlo tree search strategy, which exploits the value function of a pre-trained
goal-based policy.

% Hierarchical Reinforcement Learning
\textbf{Reinforcement learning as a global policy.}
Combining local action sequences with specific goals in order to achieve a more general goal is the core idea of
HRL~\citep{sutton1999between,precup2000temporal,konidaris2009skill}.
Notably, among recent works, \citet{kulkarni2016hierarchical} define a bi-level hierarchical policy, using a
DQN~\citep{mnih2015human} agent to select high-level goals, that define options which make use of a low-level goal-based
DQN agent.
\citet{nachum2018data} specializes this idea to the case when the lower-level policy learns to achieve goals that
encode relative changes to the current state.
\citet{levy2019learning} couples HER with a three-level hierarchy into an architecture called Hierarchical Actor-Critic
(HAC).
\citet{mcclinton2021hac} enhance HAC with a separate higher-level goal generator which drives the exploration process
during learning.
Overall, these approaches aim at designing a global neural-network-based controller, able to solve the tasks at hand.
DIAYN~\cite{eysenbach2018diversity} learn a skill-conditioned policy, by maximising the mutual information between the 
visited states and the given skill (an integer.).
Then, they use it as a local policy, by computing the executed skills using a global policy with discrete actions.
The global actions are given to the local policy to compute the hierarchical agent's action.

% Options-based Reinforcement Learning

\textbf{Graph-based global policies.}
% Reachability-graph Reinforcement Learning
Some methods store explicitly these links between sub goals by constructing a reachability graph (which encodes
transitions between abstractions of states, using abstractions of actions).
In turn, this graph can be used for higher-level goal-based planning.
\citet{savinov2018semi} build this graph by randomly exploring the environment, and add a node for every encountered
state, which yields a very dense graph.
For a given goal, a shortest path in this graph is computed.
Then a sequence of landmark sub goals is extracted so that each landmark is far enough from the previous one according
to a pre-trained neural network.
\citet{eysenbach2019search} introduce Search on the Replay Buffer (SoRB), which supposes the availability of a replay
buffer of states and defines a graph where each state in a random subset of the replay buffer is a node.
Then it uses the goal-reaching policy's value function to estimate edge weights between these nodes and finds a shortest
path of state waypoints to the goal.
\citet{huang2019mapping} propose a similar approach, but improve the node sampling strategy, and exploit a policy only
to reach close goals during pre-training.
SGM~\citep{emmons2020sparse} improves SoRB's results by pruning useless nodes in the graph, and edges that cannot been
traversed by the control policy.
Pruning useless nodes enables a reduction in the number of graph edges and permits a faster convergence to a
close-to-optimal graph (ie. representative of actual reachability with a minimal number of nodes and edges).
\citet{chaplot2020neural} learn a reachability graph in a robotics navigation environment.
For each new location in its graph, the agent uses its camera to estimate promising exploration directions.
\citet{aubret2021distop} and \citet{ruan2022target} incrementally grow a graph representing reachability, where nodes
are abstractions of sets of states, using a neural network as a surrogate of the similarity between states.
Similarly to the work of \citet{aubret2021distop}, PALMER~\citep{beker2022palmer} learns a projection space where the
distance between the embeddings of two states
represents the number of actions required by an optimal policy to reach one from the other.
\acrshort{dsg}~\citep{bagaria2021skill} incrementally grows a graph of options, represented as a local finite set of
goal-reaching skills in each node.
The present contribution uses the goal space as a state abstraction, and translation-invariant pre-trained policies as
action abstractions to enable planning in a graph of goals.

\textbf{Re-using policies across states / Connection with navigation tasks.}
Learning policies locally to re-use them in unexplored states has also been investigated to enable hierarchical RL.
%Other methods perform planning on re-usable skills and policy.
\citet{konidaris2007building} propose to perform skill chaining on skills that are dependent on an discrete set of
abstract agent contexts.
These skills become reusable across contexts, but the definition of the contexts are specific to the task to solve and
cannot be fully assimilated to an abstraction of navigation waypoints.
PRM-RL~\citep{faust2018prm} and RL-RRT~\citep{chiang2019rl} build a re-usable goal-conditioned policy, and use it to
navigate through a graph built using respectively PRM~\citep{kavraki1996probabilistic} and
RRT~\citep{lavalle1998rapidly}, hence providing an abstraction of navigation features into a more general framework.
Note however that none of these methods provide a generic and formal definition of policy re-useablity.

Some other global policies can be considered, but are less frequent in the literature of hierarchical agents using a 
local policy trained with reinforcement learning.
DADS~\citep{sharma2019dynamics} learns a set of diverse skills, in a way that is similar to DIAYN. 
As a difference, the skills to be executed are computed using model predictive control.



\begin{figure}%[h]
  \centering
  \includegraphics[width=\linewidth]{figures/state_of_the_art_vizualisation_2}
  \caption{representation of the hierarchical agents state of the art.}
  \label{figure:stateOfTheArt:vizualisation}
\end{figure}

% \begin{center}
% \begin{tabular}{| c | c | c |}
%  Algorithm & global policy & control policy \\
%  SPTM & Graph & RL \\
%  SORB & Graph & RL \\
%  SGM & Graph & RL \\
%  RL-RRT & Graph & RL \\
%  PRM-RL & Graph & RL \\
%  NTSlam & Graph & RL \\
%  DisTop & Graph & RL \\
%  STC & Graph & RL \\
%  RGL & Graph & RL \\
%  DSG & Graph & Options \\
%  BPO &  & Options \\
%  HAC & RL & RL \\
%  HIRO & RL & RL \\
%  DADS & MPC & Skills \\
%   &  &  \\
%   &  &  \\
%   &  &  \\
%   &  &  \\
%   &  &  \\
%   &  &  \\
%   &  &  \\
%   &  &  \\
%   &  &  \\
%   &  &
% \end{tabular}
% \end{center}
